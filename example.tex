% Changing book to article will make the footers match on each page,
% rather than alternate every other.
%
% Note that the article class does not have chapters.
\documentclass[10pt,% font size
  a4paper,% paper size
  twoside,% different margin for even and odd page
  openany% chapters start on the same page
  ]{book}

\usepackage[bg-a4]{dsa} % Options: bg-a4, bg-letter, bg-full, bg-print, bg-none.
\usepackage[ngerman]{babel}
\usepackage[utf8]{inputenc}


\title{The Dark \LaTeX}
\author{David}
\date{\thedate}

% Start document
\begin{document}

\makeatletter

% Select the language
\selectlanguage{ngerman}

% Set text font
\fontfamily{ppl}\selectfont

% Your content goes here

\frontmatter

\titlebackgroundimg{}
\lipsum[1] % filler text
\clearpage

%Inhaltsverzeichnis

\begingroup
\let\clearpage\relax
\tocbackgroundimg{}

\tableofcontents
\endgroup
\clearpage

\mainmatter{}

% Comment this out if you're using the article class.
\chapter{Chapter 1: The Dark \LaTeX}

\begin{multicols}{2}
\lipsum[1]

\section{Main Section}
\lipsum[1] % filler text

\subsection{Fun with boxes}
\subsubsection{Even more fun!}

\begin{commentbox}{\commentquad}
  \lipsum[1]
\end{commentbox}

\subsection{Quotebox}

\begin{quotebox}
  As you approach this template you get a sense that the blood and tears of many generations went into its making. A warm feeling welcomes you as you type your first words.
\end{quotebox}

\newpage % Acts as columbreak because of twocolumn option; for pagebreak use \clearpage

\begin{dsatable}[tabularx={XXX}, title=Nice table]
    \dsaheader{tableheadercolor}
    \textbf{Table head} & \textbf{Table head} & \textbf{Table head}\\
    \hline
    Some value  & Some value & Some value \\
    \hline
    Some value  & Some value & Some value
\end{dsatable}


\begin{paperbox}{\paperimagequad}
\begin{dsatable}[tabularx={XXX}, title=Nice table inside box]
    \dsaheader{tableheadercolor}
    \textbf{Table head} & \textbf{Table head} & \textbf{Table head}\\
    \hline
    Some value  & Some value & Some value \\
    \hline
    Some value  & Some value & Some value
\end{dsatable}
\end{paperbox}

\lipsum[1]

\begin{paperbox}{\paperimagequad}
  \lipsum[1]
\end{paperbox}

\end{multicols}
\begin{paperbox}{\paperimagequad}
  \begin{multicols}{2}
    \begin{dsalist}
      \item \lipsum[1]
      \item \lipsum[1]
      \item \lipsum[1]
    \end{dsalist}
  \end{multicols}
\end{paperbox}
\begin{multicols}{2}

% You can optionally not include the background by saying
% begin{monsterboxnobg}
\begin{monsterbox}{Monster Foo}
  \textit{Small metasyntactic variable (goblinoid), neutral evil}\\
  %\hline
  \basics[%
  armorclass = 12,
  hitpoints  = 16 (3d8 + 3),
  speed      = 50 ft
  ]
  %\hline
  \stats[
    STR = \stat{12}, % This stat command will autocomplete the modifier for you
    DEX = \stat{7}
  ]
  %\hline
  \details[%
  % If you want to use commas in these sections, enclose the
  % description in braces.
  % I'm so sorry.
  languages = {Common Lisp, Erlang},
  ]
  %\hline \\[1mm]
  \begin{monsteraction}[Monster-super-powers]
    This Monster has some serious superpowers!
  \end{monsteraction}
  \monstersection{Actions}
  \begin{monsteraction}[Generate text]
    This one can generate tremendous amounts of text! Though only when it wants to.
  \end{monsteraction}

  \begin{monsteraction}[More actions]
    See, here he goes again! Yet more text.
  \end{monsteraction}
\end{monsterbox}

\section{Icons}

%\dsaEasyHard{\lipsum[1]}{\lipsum[1]}

\dsabauer{}

\dsaspringer{}

\dsakoenig{}

\dsaturm{}

\section{Liturgien}

\begin{liturgybox}{Bann des Lichts}
\liturgy[%
  probe={MU/KL/CH},
  wirkung={Um den Geweihten herum bildet sich eine Kugel aus Dunkelheit mit einem Durchmesser von QS x 3 Schritt. Pro QS erschweren sich die Sichtverhältnisse um eine Stufe. Natürliche und magische Lichtquellen können die Dunkelheit nicht erhellen. Bei karmalen Lichtquellen entscheidet die höhere QS (wie bei einer Vergleichsprobe), ob das Licht zu sehen ist oder nicht: Hierbei gilt alles oder nichts: Das Licht wird nicht um die QS der Liturgie gedämpft, sondern die höhere QS entscheidet darüber ob Licht zu sehen ist oder nicht. Für den Geweihten werden die Sichtverhältnisse durch die Liturgie nicht erschwert. Der Geweihte muss vor dem Wirken der Liturgie entscheiden, ob die Zone der Dunkelheit an Ort und Stelle verbleiben oder sich mit ihm als Zentrum bewegen soll.},
  liturgiedauer={4 Aktionen},
  kosten={16 KaP (Aktivierung) + 8 KaP pro 5 Minuten},
  reichweite={selbst},
  wirkungsdauer={aufrechterhaltend},
  ziel={Zone},
  verbreitung={Boron (Tod und Traum), Phex (Schatten)},
  steigerung={B}]
\end{liturgybox}


\end{multicols}
\chapter{Chapter 2: The Dark Eye}
\begin{multicols}{2}

\section{Next Page}
\lipsum[1] % filler text

\pagebreak

\lipsum[1] % text

\pagebreak

\lipsum[1] % text

\pagebreak
\lipsum[1] % text

\pagebreak

\lipsum[1] % text

\pagebreak

\lipsum[1] % text

\end{multicols}

\backmatter{}

\lastpagebackgroundimg{}

\lipsum[1] % text

% End document
\end{document}
